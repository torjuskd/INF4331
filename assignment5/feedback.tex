\documentclass[a4paper]{article}

% Import some useful packages
\usepackage[margin=0.5in]{geometry} % narrow margins
\usepackage[utf8]{inputenc}
\usepackage[english]{babel}
\usepackage{hyperref}
\usepackage{minted}
\usepackage{amsmath}
\usepackage{xcolor}
\usepackage{graphicx}
\usepackage{underscore}
\definecolor{LightGray}{gray}{0.95}

\title{Peer-review of assignment 5 for \textit{INF3331-torjuskd}}
\author{Thomas Ballo, thomagb, {thomagb@student.matnat.uio.no} \\
 		Atle Nærum Eriksen, atlene, {atlene@ifi.uio.no} \\
		Gudrun Sigfusdottir, gudrunsi, {gudrunsi@student.matnat.uio.no}}

\begin{document}
\maketitle

\section{Review}\label{sec:review}

The program was tested using python3 on Linux and OS X.

%%%%%%%%%%%%%%%%%%%%%%%%%%%%%%%%%%%%%%%%%%%%%%%%%%%%%%%%%%
\subsection*{General feedback}
The assignment is well conducted and all in all a good job. A README.md file which explains how to run the different programs would have been appreciated.

Both Python programs are well structured with good use of functions, making it easier to read the code and understand the functionality. Good use of docstrings the functions to explain what they do.

\subsubsection*{Notes for improvement}
\begin{itemize}
	\item Lines shouldn't be longer than 79 characters according to the PEP 8 style guide. Some long lines like line 51 and 52 in \texttt{highlighter.py} makes it harder to read and understand the code.
\end{itemize}

%%%%%%%%%%%%%%%%%%%%%%%%%%%%%%%%%%%%%%%%%%%%%%%%%%%%%%%%%%
\subsection*{Assignment 5.1: Syntax highlighting}
The layout of the program is decent and the use of "if \_\_file\_\_ == "\_\_main\_\_"" is very welcomed. However, in general there are quite a few things that are not so great:

\begin{itemize}
\item line.strip() on line 28 is redundant (but no biggie).
\item \texttt{data = read_data} on line 47 is redundant
\item The regex on line 43 could have been compiled since it doesn't change inside the loop (but no biggie).
\item Naming. Even though you document your code, the symbol names are not sufficient to be descriptive, which partly renders the documentation moot. Examples:
	\begin{itemize}
    \item highlight(): What does it highlight? It's not an instance method, so there's no instance context to bind meaning to.
    \item syntax=False: A better name could perhaps be something along the lines of is\_syntax\_file=False? Because that's what you are actually asking.
    \item readrulefile() returns a list as per the documentation. A list of what? To use it you would need to know what it contains.
    \item Similarly, highlight() takes some arguments that also are lists, but that doesn't tell us anything about the lists.
    \item The filename argument of highlight(), does it hold a filename or a file \emph{path}? It's the latter, and the symbol name should reflect that.
    \item highlight() is said to return \textless none\textgreater, however it returns a string if a source file was given.
    \item readrulefile(), rulefilename, and more: Can easily distract your focus by being non-standard. We recommend using underscores between word boundaries. This is obviously even more important the longer the name becomes or it won't be readable.
    \item This item is debatable, but the semantic meaning of what you're doing should play an important role in how you name things. For example, the function readrulefile() is also used to load theme files in addition to syntax files. However, the theme information is not particularly rule-based. The regexes define rules for what should be searched for in a string, but color codes aren't exactly rules. You can look at it that way, though, so food for thought.
    \end{itemize}
\end{itemize}

The highlighting part of the program appeared cryptic at first sight, and from a readability perspective it could be improved. We were suspicious about whether the approach of "fixing up the colors" would work in cases where there was more than two colors to work with (since that's what the regex expects), such as a region within a region within a region. We did some tests and found out this didn't appear to be working as expected, but surprisingly the simpler test of a region within a region (standard overlapping) also failed. Even though the code seems correct in theory, there is clearly something wrong in practice, and it could be worth looking into (or rather, for a future project, use more varied test subjects).

\includegraphics[scale=0.85]{1.png}

As you can see the colors are not properly closed in this example, nor is the green region reopened after leaving the blue region.

We then added a simple regex to highlight digits and got this result:

\includegraphics[scale=0.85]{2.png}

It now becomes apparent why it's a bad idea to have compact, cryptic code instead of longer, more mundane step-by-step code, because this would probably be very hard to debug to fix.

The highlighting approach taken is to substitute the output on every regex match, which was pointed out by an instructor on Piazza to be a problematic solution. Here is an example why:

\includegraphics[scale=0.85]{3.png}

The regex first matches BC and colors that part correctly. Now, despite CD actually existing in the input the regex is unable to match it because the input has been modified in-place, and there is now "[0m" in between C and D.

Interestingly, despite these shortcomings task 5.2, 5.3 and 5.4 seem to be working just fine. It may have something to do with the way things are overlapping in those files compared to our tests, but we can't say for sure.

%%%%%%%%%%%%%%%%%%%%%%%%%%%%%%%%%%%%%%%%%%%%%%%%%%%%%%%%%%
\subsection*{Assignment 5.2:  Python syntax} \label{sec:assignment5.2}
The program works as expected, but the second theme file is missing. There is a file called python.theme2 in the directory, but it appears to be empty. python.syntax is well written and takes into consideration all possible mismatches.

The regex accounts for proper spacing between things, such that e.g. both "a=1" and "a = 1" are both valid, that's nice to see. Some of the regexes are more open than they could have been, e.g. the string "def-part" (from our test files) gets matched even though it's not a def-statement. Similarly, "class <anything here>: <some code>" matches the whole line instead of capturing only the keywords required. Perhaps this was your intention, perhaps not. It's still worth restricting the regex as much as possible so that f.ex. "I'm going to class today" isn't classified as a class definition (just an example, your program doesn't do this).

Strings match both multi-line, double-quoted, single-quoted, and even strings that have matching quotes inside the string itself, which should from that point on cancel the string (and this is colored correctly). Kudos.

The non-capturing group at the start of most of the regexes is probably not necessary.

%%%%%%%%%%%%%%%%%%%%%%%%%%%%%%%%%%%%%%%%%%%%%%%%%%%%%%%%%%
\subsection*{Assignment 5.3: Syntax for your favorite language}
Java: Again, well written and takes into consideration a lot of different aspects of the language. 

%%%%%%%%%%%%%%%%%%%%%%%%%%%%%%%%%%%%%%%%%%%%%%%%%%%%%%%%%%
\subsection*{Assignment 5.4:  Syntax for your second favorite language}
SiMPLE: Yet again, the syntax file is extensive and takes a lot into consideration. This seems to work as expected. Good job. 

%%%%%%%%%%%%%%%%%%%%%%%%%%%%%%%%%%%%%%%%%%%%%%%%%%%%%%%%%%
\subsection*{Assignment 5.5: Superdiff}
The program works as expected and is well documented. It is quite easy to read and the number of functions is acceptable. It does seem overly complicated, but by the way it is implemented, there is no obvious way to simplify it. By treating the file-lines as arrays and then poping the stack would eliminate the need for the line1N and line2N variables and shorten the code drastically.

When no output filename is provided it would have been nice if the program printed the output to the terminal instead. Now it doesn't really do anything when no output file is provided.

The algorithm is not very smart. For instance when comparing two files where the first line has been moved to the end of the file, the output is:

\begin{verbatim}
  +b
  +c
  +d
  +e
  0a
  +
  -b
  -c
  -d
  -e
  -
\end{verbatim}

This is technically not wrong, although adding and removing the last empty line is unnecessary. But the output would have been a lot easier to read if it was:

\begin{verbatim}
- a
0 b
0 c
0 d
0 e
+ a
\end{verbatim}



%%%%%%%%%%%%%%%%%%%%%%%%%%%%%%%%%%%%%%%%%%%%%%%%%%%%%%%%%%
\subsection*{Assignment 5.6:  Coloring diff}
The coloring works as expected. 

\bibliographystyle{plain}
\bibliography{literature}

\end{document}